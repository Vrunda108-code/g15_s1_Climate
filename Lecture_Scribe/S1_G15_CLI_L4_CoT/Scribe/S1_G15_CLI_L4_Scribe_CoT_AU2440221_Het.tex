\documentclass[11pt]{article}
\usepackage[a4paper,margin=1in]{geometry}
\usepackage{amsmath,amssymb}
\usepackage{enumitem}
\usepackage{hyperref}

\title{\textbf{CSE400 – Fundamentals of Probability in Computing}\\
Lecture 4: Joint Probability and Conditional Probability}
\author{Instructor: Dhaval Patel, PhD}
\date{January 15, 2026}

\begin{document}
\maketitle

\section*{Introduction to Probability Theory}

\subsection*{Experiments, Sample Space, and Events}

\paragraph{Experiment (E)}
An experiment is a procedure that produces an outcome. The result is not known in advance but follows well-defined rules.

\textbf{Example:} Tossing a coin five times.

\paragraph{Outcome ($\omega$)}
An outcome is a single possible result of an experiment.

\textbf{Example:} One possible outcome of tossing a coin five times is HHTHT.

\paragraph{Event}
An event is a collection (set) of outcomes from the same experiment. Events are usually denoted by capital letters.

\textbf{Example:} For the coin-tossing experiment, let event $C$ be:
\[
C = \{\text{all outcomes consisting of an even number of heads}\}
\]
This means $C$ contains many outcomes, not just one.

\paragraph{Sample Space ($S$)}
The sample space is the set of all possible distinct outcomes of an experiment.

\textbf{Key properties:}
\begin{itemize}
    \item \textbf{Mutually Exclusive:} No two outcomes can occur at the same time.
    \item \textbf{Collectively Exhaustive:} One of the outcomes must occur.
\end{itemize}

\textbf{Example:} For a single coin flip:
\begin{itemize}
    \item Heads and tails are mutually exclusive.
    \item Heads and tails together exhaust all possibilities.
\end{itemize}

The sample space $S$ is the universal set for the experiment.

\section*{Types of Sample Spaces}

A sample space can be:
\begin{itemize}
    \item Discrete
    \item Countably infinite
    \item Continuous
\end{itemize}

\textbf{Examples:}
\begin{itemize}
    \item Flipping a fair coin once
    \item Rolling a cubical die
    \item Rolling two dice
    \item Flipping a coin until a tail occurs
    \item Random number generator on interval $[0,1)$
\end{itemize}

\section*{Axioms of Probability}

\paragraph{Probability}
Probability is a numerical measure of how likely an event is to occur. It is a function that maps events to numbers between 0 and 1.

\paragraph{Axioms}
\begin{enumerate}
    \item For any event $A$:
    \[
    0 \leq \Pr(A) \leq 1
    \]
    \item If $S$ is the sample space:
    \[
    \Pr(S) = 1
    \]
    \item If $A$ and $B$ are mutually exclusive ($A \cap B = \varnothing$):
    \[
    \Pr(A \cup B) = \Pr(A) + \Pr(B)
    \]
    \item For an infinite collection of mutually exclusive events $A_1, A_2, \dots$:
    \[
    \Pr\left(\bigcup_i A_i\right) = \sum_i \Pr(A_i)
    \]
\end{enumerate}

\section*{Corollary from Probability Axioms}

\paragraph{Corollary 2.1}
For a finite number $M$ of mutually exclusive events $A_1, A_2, \dots, A_M$:
\[
\Pr\left(\bigcup_i A_i\right) = \sum_i \Pr(A_i)
\]

\section*{Propositions from Probability Axioms}

\paragraph{Proposition 2.1 (Complement Rule)}
\[
\Pr(A^c) = 1 - \Pr(A)
\]

\paragraph{Proposition 2.2 (Monotonicity)}
If $A \subseteq B$, then:
\[
\Pr(A) \leq \Pr(B)
\]

\paragraph{Proposition 2.3 (Addition Rule)}
\[
\Pr(A \cup B) = \Pr(A) + \Pr(B) - \Pr(A \cap B)
\]

\paragraph{Proposition 2.4 (Inclusion--Exclusion Principle)}
\[
\Pr\left(\bigcup_{i=1}^M A_i\right)
= \sum_i \Pr(A_i)
- \sum_{i<j} \Pr(A_i \cap A_j)
+ \sum_{i<j<k} \Pr(A_i \cap A_j \cap A_k)
- \cdots
+ (-1)^{M+1} \Pr\left(\bigcap_{i=1}^M A_i\right)
\]

\section*{Assigning Probabilities}

\subsection*{Classical Approach}
Assumes all outcomes in the sample space are equally likely.

\textbf{Examples:}
\begin{itemize}
    \item Coin flip
    \item Dice roll
    \item Pair of dice
\end{itemize}

\[
\Pr(\text{Event}) = \frac{\text{Number of favorable outcomes}}{\text{Total number of outcomes}}
\]

\subsection*{Relative Frequency Approach}
\[
\Pr(A) \approx \frac{n_A}{n}
\]
where $n$ is the total number of trials and $n_A$ is the number of times $A$ occurs.

\section*{Joint Probability}

\paragraph{Definition}
\[
\Pr(A, B) = \Pr(A \cap B)
\]

\paragraph{Multiple Events}
\[
\Pr(A_1, A_2, \dots, A_M)
\]

\subsection*{Calculation Approaches}

\paragraph{Relative Frequency}
\[
\Pr(A, B) = \lim_{n \to \infty} \frac{n_{AB}}{n}
\]

\subsection*{Example: Card Deck}

\[
\Pr(A) = \frac{26}{52}, \quad
\Pr(B) = \frac{40}{52}, \quad
\Pr(C) = \frac{13}{52}
\]

\[
\Pr(A,B)=\frac{20}{52}, \quad
\Pr(A,C)=\frac{13}{52}, \quad
\Pr(B,C)=\frac{10}{52}
\]

\section*{Conditional Probability}

\paragraph{Definition}
\[
\Pr(A \mid B) = \frac{\Pr(A \cap B)}{\Pr(B)}, \quad \Pr(B) > 0
\]

\paragraph{Product Rule}
\[
\Pr(A,B) = \Pr(A \mid B)\Pr(B)
\]

\paragraph{Chain Rule}
\[
\Pr(A_1,\dots,A_M)
= \prod_{i=1}^{M} \Pr(A_i \mid A_1,\dots,A_{i-1})
\]

\subsection*{Example: Cards Without Replacement}

\[
\Pr(B \mid A) = \frac{12}{51}
\]

\subsection*{Poker Flush}

\[
\Pr(\text{Spade Flush})
= \frac{13}{52}\cdot\frac{12}{51}\cdot\frac{11}{50}\cdot\frac{10}{49}\cdot\frac{9}{48}
\]

\[
\Pr(\text{Any Flush}) = 4 \times \Pr(\text{Spade Flush})
\]

\subsection*{Missing Key Problem}

\[
\Pr(R \mid L^c) = \frac{0.4}{0.6} = \frac{2}{3}
\]

\end{document}
