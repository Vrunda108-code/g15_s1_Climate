\documentclass[11pt]{article}

\usepackage[a4paper,margin=1in]{geometry}
\usepackage{amsmath,amssymb}
\usepackage{physics}
\usepackage{hyperref}
\usepackage{setspace}

\setstretch{1.2}
\setlength{\parindent}{0pt}
\setlength{\parskip}{0.6em}

\title{\textbf{CSE400 – Fundamentals of Probability in Computing}\\
Lecture 8: Gaussian, Uniform, Exponential, and Gamma Random Variables}
\author{Instructor: Dhaval Patel, PhD}
\date{January 29, 2026}

\begin{document}

\maketitle
\hrule
\vspace{1em}

\section{Gaussian Random Variable: Definition and Properties}

\textbf{Definition}

A Gaussian (Normal) random variable is a continuous random variable whose
probability density function (PDF) is given by:
\[
f_X(x) = \frac{1}{\sqrt{2\pi\sigma^2}}
\exp\left( -\frac{(x-\mu)^2}{2\sigma^2} \right)
\]

where:
\begin{itemize}
\item $\mu$ is the mean (location parameter),
\item $\sigma^2$ is the variance (spread parameter),
\item $\sigma$ is the standard deviation.
\end{itemize}

Notation:
\[
X \sim \mathcal{N}(\mu, \sigma^2)
\]

This PDF is symmetric about $x=\mu$, which implies equal probability mass on both
sides of the mean.

\textbf{Properties}
\begin{itemize}
\item Mean: $E[X] = \mu$
\item Variance: $\text{Var}(X) = \sigma^2$
\item Symmetry: The distribution is symmetric around the mean.
\item Skewness: Zero (due to symmetry).
\item Kurtosis: Fixed value characteristic of Gaussian distributions.
\end{itemize}

The cumulative distribution function (CDF) does not have a closed-form expression
in elementary functions and is evaluated using special functions.

\section{Gaussian Random Variable: Standard Forms}

\subsection{Error Function (erf)}

The error function is defined as:
\[
\text{erf}(x) = \frac{2}{\sqrt{\pi}} \int_{0}^{x} e^{-t^2} \, dt
\]

It arises naturally when integrating the Gaussian PDF.

\subsection{Complementary Error Function (erfc)}

The complementary error function is defined as:
\[
\text{erfc}(x) = 1 - \text{erf}(x)
= \frac{2}{\sqrt{\pi}} \int_{x}^{\infty} e^{-t^2} \, dt
\]

This function is useful for tail probability calculations.

\subsection{$\Phi$-function (Standard Normal CDF)}

Let $Z \sim \mathcal{N}(0,1)$.  
The CDF of a standard normal random variable is:
\[
\Phi(x) = \frac{1}{\sqrt{2\pi}} \int_{-\infty}^{x} e^{-t^2/2} \, dt
\]

\subsection{Q-function (Gaussian Tail Function)}

The Q-function is defined as:
\[
Q(x) = \frac{1}{\sqrt{2\pi}} \int_{x}^{\infty} e^{-t^2/2} \, dt
\]

It represents the right-tail probability of a standard Gaussian random variable.

\textbf{Key Identity}
\[
Q(x) = 1 - \Phi(x)
\]

\section{Relation Between $\Phi$-function and Q-function}

For a general Gaussian random variable $X \sim \mathcal{N}(\mu,\sigma^2)$, the CDF
is evaluated by standardization:
\[
F_X(x) = \Pr(X \le x) = \Phi\left(\frac{x-\mu}{\sigma}\right)
\]

This step converts $X$ into a standard normal variable.

\textbf{Evaluating Tail Probabilities}
\[
\Pr(X > x) = Q\left(\frac{x-\mu}{\sigma}\right)
\]

\textbf{Key Identities}
\[
F_X(x) = 1 - Q\left(\frac{x-\mu}{\sigma}\right)
\]

\textbf{Interpretation}
\begin{itemize}
\item $\Phi(x)$: Area under the left tail of the Gaussian curve.
\item $Q(x)$: Area under the right tail of the Gaussian curve.
\end{itemize}

\section{Gaussian Random Variable: Example}

Given a random variable $X$ with PDF:
\[
f_X(x) = \frac{1}{\sqrt{8\pi}} e^{-x^2/8}
\]

By comparison with the standard Gaussian form, the parameters $\mu$ and
$\sigma^2$ are identified.

\textbf{Required Probabilities}
\begin{itemize}
\item $\Pr(X < 0)$ expressed using the $\Phi$-function.
\item $\Pr(X > 4)$ expressed using the $Q$-function.
\item $\Pr(|X+3| < 2)$ converted into an interval probability.
\item $\Pr(|X-2| > 1)$ converted into tail probabilities.
\end{itemize}

\textbf{Hint from lecture}
\begin{itemize}
\item Use $Q(x)$ for right-tail probabilities.
\item Use $\Phi(x)$ for left-tail probabilities.
\end{itemize}

\section{Applications of Gaussian Random Variables}

Gaussian models arise naturally in many engineering and computing contexts:
\begin{itemize}
\item Thermal noise in electronic circuits.
\item Measurement errors in sensors:
\[
\text{Measured Value} = \text{True Value} + \text{Gaussian Noise}
\]
\item Packet delay variation (jitter) in communication networks.
\end{itemize}

\section{Problem Solving: CDF Analysis Exercise}

Given CDF:
\[
F_X(x) =
\begin{cases}
0, & x < 0 \\
2x, & 0 < x < 1 \\
1, & x > 1
\end{cases}
\]

\textbf{Step 1: Find the PDF}
\[
f_X(x) = \frac{d}{dx} F_X(x) = 2x, \quad 0 < x < 1
\]

\textbf{Step 2: Mean}
\[
\mu_X = \int_{0}^{1} x(2x)\,dx = \frac{2}{3}
\]

\textbf{Step 3: Variance}
\[
\sigma_X^2 = E[X^2] - (E[X])^2
\]

\textbf{Step 4: Skewness and Kurtosis}

Derived from central moments.

\textbf{Conclusion}

The distribution is left-skewed and platykurtic.

\section{Gaussian Modeling: From Noise to Math}

Noise is modeled using a Gaussian random variable:
\[
X = \sigma Z + \mu, \quad Z \sim \mathcal{N}(0,1)
\]

This transformation maps standard randomness to a physical model.

\textbf{Key Idea}

Probability models the shape of uncertainty, not exact values.

\section{Density Estimation: Learning from Data}

In practice, the true $\mu$ and $\sigma$ are unknown.

Observed data is represented as a histogram.

The estimated PDF smooths noise into a predictive model.

\textbf{Key Concept}

We estimate probability distributions, not deterministic functions.

\section{Applications: Network Systems}

\textbf{Case Study: Packet Delay Modeling}

500 ICMP ping measurements show delay variation due to queuing and congestion.

\textbf{Engineering Goals}
\begin{itemize}
\item Estimate typical delay (mean).
\item Measure jitter (variance).
\end{itemize}

High jitter is more damaging to real-time video than constant delay.

\section{Applications in Broader Computing Contexts}

\begin{itemize}
\item Image Processing: Gaussian priors for denoising.
\item Cloud Systems: Tail latency modeling.
\item IoT \& Robotics: Sensor fusion using probabilistic filters.
\end{itemize}

\textbf{Final Insight}

Uncertainty is unavoidable and probability is the key tool to manage it.

\end{document}
