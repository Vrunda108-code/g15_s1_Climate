\documentclass[12pt]{article}
\usepackage{amsmath, amssymb}
\usepackage{geometry}
\usepackage{enumitem}
\geometry{margin=1in}

\title{Lecture Scribe Probability}
\date{}

\begin{document}

\maketitle

\section*{CSE400 – Fundamentals of Probability in Computing}

\textbf{Lecture L11–L12: Transformation of Random Variables} \\
Prof. Dhaval Patel, PhD \\
Associate Professor, Computer Science and Engineering \\
SEAS – Ahmedabad University \\
Date: February 10–12, 2026

\section*{Outline}

\begin{enumerate}
    \item Transformation of Random Variables \\
    Learning transformation techniques for random variables.
    
    \item Function of Two Random Variables \\
    Joint transformations and derived distributions.
    
    \item Illustrative Example \\
    Detailed derivation for the case.
\end{enumerate}

\newpage

\section*{1️⃣ Transformation of Random Variables}

We consider a transformation:

\[
Y = g(X)
\]

Our goal is to find the distribution (CDF and PDF) of $Y$, given that the distribution of $X$ is known.

\subsection*{Step–V: CDF of $Y$}

From the slide:

\[
F_Y(y) = P(Y \le y)
\]

Since $Y = g(X)$,

\[
F_Y(y) = P(g(X) \le y)
\]

The probability is rewritten in terms of $X$, because the PDF of $X$ is known.

The slide shows transformation using:

\[
F_Y(y) = 1 - F_X(\cdot)
\]

depending on the monotonicity of the function.

Thus the method used in the slides:

\begin{enumerate}
    \item Express event $Y \le y$
    \item Rewrite in terms of $X$
    \item Use known CDF of $X$
    \item Differentiate to get PDF of $Y$
\end{enumerate}

Finally:

\[
f_Y(y) = \frac{d}{dy} F_Y(y)
\]

\newpage

\section*{2️⃣ Example: Uniform Random Variable}

Given:

\[
X \sim \text{Uniform}(-1,1)
\]

\[
f_X(x) = \frac{1}{2}, \quad -1 < x < 1
\]

Find:

\[
Y = g(X) = \sin\left(\frac{\pi}{2}X\right)
\]

\subsection*{Step 1: Define transformation}

\[
F_Y(y) = P(Y \le y)
\]

\[
= P\left(\sin\left(\frac{\pi}{2}X\right) \le y\right)
\]

Rewrite inequality in terms of $X$, then use uniform PDF.

Final PDF (as shown in slide):

\[
f_Y(y) = \frac{1}{\pi \sqrt{1-y^2}}, \quad -1 < y < 1
\]

This result comes from differentiating the transformed CDF and using the Jacobian term that appears after inversion.

\newpage

\section*{3️⃣ Function of Two Random Variables}

Now consider:

\[
Z = g(X,Y)
\]

Case:

\[
Z = X + Y
\]

We start with CDF:

\[
F_Z(z) = P(Z \le z)
\]

\[
= P(X + Y \le z)
\]

This defines a region in the $(x,y)$-plane.

Thus:

\[
F_Z(z) = \iint_{x+y \le z} f_{X,Y}(x,y)\, dx\, dy
\]

If $X$ and $Y$ are independent:

\[
f_{X,Y}(x,y) = f_X(x) f_Y(y)
\]

To obtain PDF from CDF:

\[
f_Z(z) = \frac{d}{dz} F_Z(z)
\]

\subsection*{Leibniz Rule (Used in Slides)}

If

\[
G(z) = \int_{a(z)}^{b(z)} h(x,z)\, dx
\]

then

\[
\frac{dG(z)}{dz}
=
h(b(z),z)\frac{db}{dz}
-
h(a(z),z)\frac{da}{dz}
+
\int_{a(z)}^{b(z)} \frac{\partial h(x,z)}{\partial z}\, dx
\]

This rule is applied to differentiate the CDF integral.

\newpage

\section*{Convolution Result}

After applying Leibniz rule and simplification:

\[
f_Z(z) = \int_{-\infty}^{\infty} f_X(x) f_Y(z-x)\, dx
\]

This is the convolution formula (as derived in slides).

\section*{Example: Exponential Random Variables}

Given independent RVs:

\[
f_X(x) = \lambda e^{-\lambda x}, \quad x \ge 0
\]

\[
f_Y(y) = \lambda e^{-\lambda y}, \quad y \ge 0
\]

Find:

\[
Z = X + Y
\]

Using convolution:

\[
f_Z(z)
=
\int_0^z \lambda e^{-\lambda x}
\lambda e^{-\lambda (z-x)} dx
\]

\[
=
\lambda^2 e^{-\lambda z} \int_0^z dx
\]

\[
=
\lambda^2 z e^{-\lambda z}
\]

This matches the derived expression in the slides.

\section*{Additional Example}

\[
Z = X - Y
\]

Rewrite:

\[
F_Z(z) = P(X - Y \le z)
\]

\[
= P(X \le z + Y)
\]

And integrate over joint PDF accordingly.

\newpage

\section*{Summary of Lecture Content}

\begin{enumerate}
    \item Transformation of single RV using CDF method
    \item Deriving PDF via differentiation
    \item Handling monotonic functions
    \item Transformation of two RVs
    \item Derivation of convolution formula
    \item Application to:
    \begin{itemize}
        \item Uniform transformation
        \item Sum of independent RVs
        \item Exponential case
        \item Difference case
    \end{itemize}
\end{enumerate}

\end{document}
