\documentclass[12pt]{article}
\usepackage{amsmath, amssymb}
\usepackage[a4paper,margin=1in]{geometry}

\begin{document}

\title{CSE400 — Fundamentals of Probability in Computing\\
Lecture L11/L12: Transformation of Random Variables}
\date{February 10–12, 2026}
\maketitle

\section*{Outline (as given)}

Transformation of Random Variables\\
Learning transformation techniques for random variables.

Function of Two Random Variables\\
Joint transformations and derived distributions.

Illustrative Example\\
Detailed derivation for the case:
\[
Z = X + Y
\]

L12\_S1\_A (2)

\section*{1. Transformation of a Single Random Variable}

Let $X$ be a known random variable with known PDF.

Let a new random variable be defined as
\[
Y = g(X)
\]

The goal is to find:

CDF of $Y$:
\[
F_Y(y)
\]

PDF of $Y$:
\[
f_Y(y)
\]

\subsection*{Step-by-Step Method (as presented)}

\textbf{Step 1 — Start from the CDF definition}
\[
F_Y(y)=\Pr(Y\le y)
\]

\textbf{Step 2 — Substitute $Y=g(X)$}
\[
F_Y(y)=\Pr(g(X)\le y)
\]

\textbf{Step 3 — Convert inequality into an equivalent condition on $X$}

This produces an interval (or union of intervals) in terms of $X$:
\[
=\Pr(X\in \text{corresponding region})
\]

\textbf{Step 4 — Express using the CDF of $X$}
\[
F_Y(y)=F_X(\text{boundary})
\]

or equivalently
\[
F_Y(y)=1-F_X(\text{boundary})
\]

depending on monotonicity.

\textbf{Step 5 — Differentiate to obtain the PDF}
\[
f_Y(y)=\frac{d}{dy}F_Y(y)
\]

This yields the required PDF of $Y$.

These steps are explicitly shown in the slides as:

Write CDF of $Y$

Replace $Y$ by function of $X$

Convert probability to $X$-domain

Use known CDF/PDF of $X$

Differentiate to obtain $f_Y(y)$

L12\_S1\_A (2)

\section*{Worked Example 1 (Single RV)}

Given

$X$ is uniformly distributed on $(-1,1)$

Hence,
\[
f_X(x)=\frac{1}{2},\quad -1<x<1
\]

Define
\[
Y=\sin(\pi X)
\]

Required

Find the PDF of $Y$.

\subsection*{Solution (exact lecture flow)}

\textbf{Step 1 — Start with CDF of $Y$}
\[
F_Y(y)=\Pr(Y\le y)
\]

\textbf{Step 2 — Substitute transformation}
\[
=\Pr(\sin(\pi X)\le y)
\]

\textbf{Step 3 — Convert inequality to bounds on $X$}

This produces inverse–sine type limits on $X$ (shown graphically/analytically in slides).

\textbf{Step 4 — Use PDF of uniform $X$}

Since
\[
f_X(x)=\frac{1}{2}
\]

the probability becomes an integral over the corresponding $X$-interval.

\textbf{Step 5 — Differentiate CDF to obtain PDF}

Final result (as written in slides):
\[
f_Y(y)=\frac{1}{\pi\sqrt{1-y^2}},\quad |y|<1
\]

and
\[
f_Y(y)=0 \text{ otherwise.}
\]

L12\_S1\_A (2)

\section*{2. Function of Two Random Variables}

Let two random variables $X,Y$ be given.

Define a new RV:
\[
Z=g(X,Y)
\]

Goal: find CDF/PDF of $Z$.

\subsection*{General Definition}

\[
F_Z(z)=\Pr(Z\le z)=\Pr(g(X,Y)\le z)
\]

This is evaluated by integrating the joint PDF over the appropriate region:
\[
F_Z(z)=\iint_{g(x,y)\le z} f_{X,Y}(x,y)\,dx\,dy
\]

L12\_S1\_A (2)

\section*{Illustrative Example: $Z=X+Y$}

\textbf{Step 1 — Write CDF}
\[
F_Z(z)=\Pr(X+Y\le z)
\]

\textbf{Step 2 — Convert to double integral}
\[
F_Z(z)=\iint_{x+y\le z} f_X(x)f_Y(y)\,dx\,dy
\]

(Independence is assumed in the lecture examples.)

\textbf{Step 3 — Express limits explicitly}
\[
=\int_{-\infty}^{\infty}\int_{-\infty}^{z-x} f_X(x)f_Y(y)\,dy\,dx
\]

\textbf{Step 4 — Differentiate to obtain PDF}

Using Leibniz rule for differentiation under the integral sign (explicitly shown):

If
\[
G(z)=\int h(x,z)\,dx
\]

then
\[
\frac{d}{dz}G(z)=\int \frac{\partial}{\partial z}h(x,z)\,dx
\]

Applying this,
\[
f_Z(z)=\frac{d}{dz}F_Z(z)
\]

leads to the convolution form:
\[
f_Z(z)=\int_{-\infty}^{\infty} f_X(x)f_Y(z-x)\,dx
\]

L12\_S1\_A (2)

\section*{Example: Exponential Random Variables}

Given:

$X,Y$ are exponential RVs

PDFs shown in slides as:
\[
f_X(x)=\lambda e^{-\lambda x},\quad x\ge0
\]
\[
f_Y(y)=\lambda e^{-\lambda y},\quad y\ge0
\]

Define:
\[
Z=X+Y
\]

Using convolution:
\[
f_Z(z)=\int_0^z \lambda e^{-\lambda x}\lambda e^{-\lambda(z-x)}\,dx
\]

Step-by-step (as written)

Substitute PDFs

Combine exponentials

Take constants outside

Integrate over $x$ from $0$ to $z$

Final expression in slides:
\[
f_Z(z)=\lambda^2 z e^{-\lambda z},\quad z\ge0
\]

and
\[
f_Z(z)=0 \text{ otherwise.}
\]

L12\_S1\_A (2)

\section*{Additional Slide Examples (listed)}

Linear combinations such as:
\[
Z=X-2Y
\]

CDF written as:
\[
F_Z(z)=\Pr(X-2Y\le z)
\]

and converted into joint-probability regions in the $x$-$y$ plane, following the same steps:

Write CDF

Express inequality

Set integration region

Integrate joint PDF

L12\_S1\_A (2)

\section*{Key Exam Takeaways (from slides)}

Always start with CDF definition.

Substitute transformation explicitly.

Convert probability into integration region.

Use known PDF(s).

Apply Leibniz rule when differentiating integrals.

For $Z=X+Y$, final PDF is obtained via convolution.

For single-RV transformations, use CDF → differentiate.

\end{document}
