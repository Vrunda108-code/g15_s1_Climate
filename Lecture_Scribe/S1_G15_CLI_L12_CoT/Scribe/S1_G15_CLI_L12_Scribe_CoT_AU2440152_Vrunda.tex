\documentclass{article}
\usepackage{amsmath,amssymb,geometry}
\geometry{margin=1in}
\begin{document}

\section*{CSE400: Fundamentals of Probability in Computing}

\subsection*{Lecture 12 — Transformation of Random Variables}

\hrulefill

\section*{Slide 1 — Lecture Title}

\subsection*{Transformation of Random Variables}

Given:

\begin{itemize}
\item Continuous Random Variable (CRV) (X)
\item Known PDF (f\_X(x)) and CDF (F\_X(x))
\item New Random Variable:
\end{itemize}

$$
Y = g(X)
$$

Goal:

Find:

$$
f_Y(y), \quad F_Y(y)
$$

Also transformations:

$$
Z = g(X,Y)
$$

Examples shown:

$$
X+Y, \quad X-Y, \quad X/Y, \quad X \cdot Y
$$

\hrulefill

\section*{Slide 2 — Outline}

1. \textbf{Transformation of Random Variables}

Learning of transformation techniques for random variables.

2. \textbf{Function of Two Random Variables}

Joint transformations and derived distributions.

3. \textbf{Illustrative Example}

Detailed derivation for:

$$
Z = X + Y
$$

\hrulefill

\section*{Slide 3 — Transformation of RVs (Single Variable)}

Assume:

$$
Y = g(X)
$$

where (g(\cdot)) is monotonic and invertible.

\subsection*{Step-1: CDF Method}

$$
F_Y(y) = P(Y \le y)
$$

Substitute:

$$
= P(g(X) \le y)
$$

Using inverse:

$$
= P(X \le g^{-1}(y))
$$

Therefore:

$$
F_Y(y) = F_X(g^{-1}(y))
$$

\hrulefill

Differentiate:

$$
f_Y(y) = \frac{d}{dy} F_Y(y)
$$

$$
= \frac{d}{dy} F_X(g^{-1}(y))
$$

Apply chain rule:

$$
f_Y(y)
= f_X(g^{-1}(y)) \cdot \frac{d}{dy} g^{-1}(y)
$$

Equivalently written:

$$
f_Y(y)
= f_X(x) \left|\frac{dx}{dy}\right|_{x=g^{-1}(y)}
$$

\hrulefill

\section*{Slide 4 — Transformation Cases}

\subsection*{Case S1 (Monotone decreasing)}

$$
F_Y(y) = P(Y \le y)
$$

$$
= P(X \ge g^{-1}(y))
$$

$$
= 1 - F_X(g^{-1}(y))
$$

\hrulefill

\subsection*{Case S2 — Differentiate}

$$
f_Y(y) = - f_X(g^{-1}(y)) \frac{d}{dy}[g^{-1}(y)]
$$

or

$$
f_Y(y)=\frac{f_X(x)}{|dy/dx|}\Big|_{x=g^{-1}(y)}
$$

(PDF of (X) is assumed known.)

\hrulefill

\subsection*{Case S3}

Change limits according to transformation.

\hrulefill

\section*{Slide 5 — Example (Single RV Transformation)}

Given:

$$
X \sim \text{Uniform}(-1,1)
$$

$$
f_X(x)=
\begin{cases}
\frac12, & -1<x<1\\
0, & \text{otherwise}
\end{cases}
$$

Transformation:

$$
Y=\sin\left(\frac{\pi X}{2}\right)
$$

\subsection*{Step-1: Inverse}

$$
x=\frac{2}{\pi}\sin^{-1}(y)
$$

\subsection*{Step-2: Derivative}

$$
\frac{dx}{dy}
= \frac{2}{\pi}\frac{1}{\sqrt{1-y^2}}
$$

\subsection*{Step-3: PDF}

$$
f_Y(y)=f_X(x)\left|\frac{dx}{dy}\right|
$$

Substitute:

$$
=\frac12 \cdot \frac{2}{\pi}\frac{1}{\sqrt{1-y^2}}
$$

$$
f_Y(y)=\frac{1}{\pi\sqrt{1-y^2}},\quad -1<y<1
$$

Limits shown:

$$
x=-1 \Rightarrow y=-1,\quad x=1 \Rightarrow y=1
$$

\hrulefill

\section*{Slide 6 — Function of Two Random Variables}

Let:

$$
Z=X+Y
$$

Tasks listed:

1. Find PDF (f\_Z(z))

2. If (X,Y) independent

3. If (X,Y \sim N(0,1)), show:

$$
Z \sim N(0,2)
$$

4. If exponential RVs with parameter (\lambda), find (f\_Z(z))

\hrulefill

\section*{Slide 7 — CDF of Sum}

$$
F_Z(z)=P(Z\le z)
$$

$$
= P(X+Y \le z)
$$

Using joint density:

$$
F_Z(z)=\iint_{x+y\le z} f_{XY}(x,y),dx,dy
$$

Integral form shown:

$$
\int_{-\infty}^{\infty}\int_{-\infty}^{z-y} f_{XY}(x,y),dx,dy
$$

\hrulefill

\section*{Slide 8 — Leibnitz Rule}

Let:

$$
G(z)=\int_{a(z)}^{b(z)} h(z,y),dy
$$

Then:

$$
\frac{dG(z)}{dz}
= \frac{db(z)}{dz}h(z,b(z))

* \frac{da(z)}{dz}h(z,a(z))

- \int_{a(z)}^{b(z)}\frac{\partial}{\partial z}h(z,y),dy
$$

\hrulefill

Differentiate CDF:

$$
f_Z(z)=\frac{d}{dz}F_Z(z)
$$

$$
=\int_{-\infty}^{\infty} f_{XY}(z-y,y),dy
$$

Equivalent expression:

$$
f_Z(z)=\int_{-\infty}^{\infty} f_{XY}(x,z-x),dx
$$

(Result labelled (i) in slide.)

\hrulefill

\section*{Slide 10 — Convolution (Independent Case)}

If (X) and (Y) independent:

$$
f_{XY}(x,y)=f_X(x)f_Y(y)
$$

Therefore:

$$
f_Z(z)=\int_{-\infty}^{\infty} f_X(x),f_Y(z-x),dx
$$

Named:

\textbf{Convolution Integral}

\subsection*{Gaussian Example}

Given:

$$
X\sim N(0,1), \quad Y\sim N(0,1)
$$

$$
f_X(x)=\frac{1}{\sqrt{2\pi}}e^{-x^2/2},\quad
f_Y(y)=\frac{1}{\sqrt{2\pi}}e^{-y^2/2}
$$

Substitute into convolution:

$$
f_Z(z)=\int_{-\infty}^{\infty}
\frac{1}{2\pi}
e^{-(x^2+(z-x)^2)/2},dx
$$

\hrulefill

\section*{Slide 11 — Completing Gaussian Result}

Integral manipulation shown on slide:

$$
=\frac{1}{\sqrt{2\pi}\sqrt{2}}e^{-z^2/(2\cdot2)}
$$

Hence:

$$
Z \sim N(0,2)
$$

\hrulefill

\section*{Slide 12 — Exponential RVs Example}

Given:

$$
f_X(x)=\lambda e^{-\lambda x},\quad x>0
$$

$$
f_Y(y)=\lambda e^{-\lambda y},\quad y>0
$$

Use convolution:

$$
f_Z(z)=\int_{-\infty}^{\infty} f_X(x)f_Y(z-x),dx
$$

Substitute:

$$
=\int_0^z \lambda e^{-\lambda x}\cdot \lambda e^{-\lambda(z-x)}dx
$$

$$
=\lambda^2 e^{-\lambda z}\int_0^z dx
$$

$$
=\lambda^2 z e^{-\lambda z},\quad z>0
$$

Otherwise:

$$
0
$$

\hrulefill

\section*{Slide 13 — Difference of RVs}

Example:

$$
Z=X-Y
$$

CDF:

$$
F_Z(z)=P(Z\le z)=P(X-Y\le z)
$$

Integral form shown:

$$
\iint f_{XY}(x,y),dx,dy
$$

PDF obtained by:

$$
f_Z(z)=\frac{d}{dz}F_Z(z)
$$

Note written:

Same procedure as (Z=X+Y).

\hrulefill

\section*{Slide 14 — Ratio and Radial Transformation}

\subsection*{Ratio}

$$
Z=\frac{X}{Y}
$$

CDF setup:

$$
F_Z(z)=P\left(\frac{X}{Y}\le z\right)
$$

Cases indicated:

\begin{itemize}
\item (Y>0)
\item (Y<0)
\end{itemize}

\hrulefill

\subsection*{Radial Example}

$$
R=\sqrt{X^2+Y^2}
$$

Goal stated:

$$
f_R(r)=?
$$

(No further derivation shown on slide.)

\hrulefill

\section*{End of Lecture 12 Scribe}

\end{document}
